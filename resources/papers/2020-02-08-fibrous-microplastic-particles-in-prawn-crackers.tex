%! Author = Frank
%! Date = 18/01/2020

% Preamble
\documentclass[11pt]{article}

% Packages
\usepackage{siunitx}
\usepackage{array}

\title{Fibrous microplastic particles in prawn crackers}
\date{2020, February 8}
\author{Frank Prins}

% Document
\begin{document}

    \maketitle

    \section{Abstract}
    Microplastics have been found in multiple studies into it's presence in sea food.
    I hypothesize that processed food, specifically prawn crackers, might also contain micro plastics.
    To test this hypothesis, the present study has investigated the presence of fibrous micro plastic candidates in 12 different brands of prawn crackers bought in supermarkets in the Netherlands.
    The number of fibrous microplastic candidates in each brand of prawn crackers was ranging from 2/gram to 12/gram, averaging at 6/gram.
    This result indicates that processed foods derived from sea food can be contaminated by micro plastics too.

    \section{Introduction}
    Microplastics have become a concern because of the global impact in marine environments~\cite{andrady2011microplastics,woodall2014deep}.
    Recently, multiple studies have also shown presence of microplastics in salt~\cite{yang2015microplastic,karami2017presence}, sea water~\cite{woodall2014deep,van2013microplastic} and sea food~\cite{efsa2016presence}, but not a lot of research has been done into the presence of microplastics in processed food.
    As sea food is known to contain microplastics~\cite{efsa2016presence}, I hypothesized that processed food, specifically prawn crackers might also contain micro plastics.
    To test this hypothesis, I collected 12 different brands of prawn crackers from different supermarkets throughout the netherlands.
    The presence of fibrous micro plastic candidates was measured.

    \section{Methods}
        \subsection{Collection of Prawn crackers}
            12 brands of prawn crackers were collected in supermarkets in the Netherlands during january 2020.
            One brand of vegetarian cassava crackers was tested to correct for any potential procedural contamination.
        \subsection{Quality Control}
            To avoid contamination, samples were covered when not in use, and experiments were finished as soon as possible.
            All containers were rinsed two times with 70\% ethanol denatured with 5\% methanol.
            All filter paper was observed through the Bresser researcher trino microscope at 100x magnification before the experiment to make sure no potentially contaminated paper was used.
        \subsection{Sample preparation}
            \SI{1}{\gram} samples were ground using a mortar and pestle and transferred to \SI{25}{\milli\meter} beakers.
            \SI{20}{\milli\meter} of 70\% ethanol denatured with 5\% methanol was measured using a graduated pipette and added to the beaker.
            A glass rod was used to stir.
        \subsection{Filtration}
            The samples were decanted and directly poured off into a hirsch funnel.
            Subsequently, the samples were filtered using \SI{15}{\milli\meter} \SI{11}{\micro\metre} filter paper and a vacuum system.
            The filter papers were then placed on a microscope slide.
        \subsection{Visual observation under a microscope}
            A visual assessment was performed to identify fibrous microplastic candidates using a Bresser researcher trino microscope at 100x magnification.
            Only colored fibrous plastic candidates were counted, non fibrous or colorless microplastic candidate identifications were disregarded.

    \section{Results}
        Although varying by number, all prawn cracker brands contained fibrous micro plastic candidates, averaging at a little over 6 per gram and ranging from almost 2 per gram to almost 12 per gram.
        Only one sample didn't contain any fibrous candidates, but the other samples for the same brand did. (Table~\ref{table:fibmicplascount})
        Contamination with airborne (microplastic) fibers and cross contamination was prevented as the control samples didn't contain fibrous plastic candidates.

    \section{Discussion}
        In this study, the fibrous microplastic candidate presence in 12 brands of prawn crackers was measured.
        In the control samples, no evidence of fibrous candidates was found.
        Identification of plastic candidates was done visually using a microscope, and colorless fibers were disregarded because of the similarities to the filter paper fibers.
        Because of this, and because of not taking into account non fibrous candidates as no means to distinguish them from non plastic particles were available, it is reasonable to assume the actual number of plastic particle candidates is higher.
        Currently, studies have already found microplastics in sea water~\cite{woodall2014deep,van2013microplastic} and sea food~\cite{efsa2016presence}.
        Even though the prawn crackers consistently contained micro fibrous plastic candidates and the vegetarian cassava based crackers could be used reliably as a control sample containing no fibrous plastic candidates, nothing can be said about the origin of the plastic particle candidates found.
        The prawn crackers were specifically chosen as a subject of research because they were made of prawns, which previously was shown to contain micro plastics~\cite{abbasi2018microplastics} and were thus a likely target to find micro plastics in processed foods.
        Other studies showed synthetic fibers in atmospheric fallout and indoor environments~\cite{dris2017first}, possibly originating in synthetic clothes and furniture~\cite{dris2017first}.
        Salt - one of the ingredients of the prawn crackers - is also a possible origin, shown to contain varying numbers of plastic particles in different studies~\cite{yang2015microplastic,karami2017presence}
        The plethora of possible origins makes it impossible to determine if the prawn as the ingredient is the reason for the presence of these fibrous microplastic candidates in prawn crackers.
        Further research should be done to determine the composition of the plastic candidates and into the presence of other non fibrous plastics in these prawn crackers.

    \begin{table}[htbp]
        \centering
        \caption{Fibrous microplastic candidate count in prawn crackers}
        \label{table:fibmicplascount}
        \begin{tabular}{>{\bfseries}ccccc}
            \hline
            Sample (no) & Run 1 & Run 2 & Run 3 & Ctrl\\
            \hline
            1 & 3 & 7 & 4 & 0 \\
            2 & 17 & 7 & 10 & 0 \\
            3 & 7 & 2 & 2 & 0 \\
            4 & 2 & 3 & 8 & 0 \\
            5 & 6 & 5 & 4 & 0 \\
            6 & 7 & 6 & 6 & 0 \\
            7 & 4 & 6 & 4 & 0 \\
            8 & 2 & 2 & 1 & 0 \\
            9 & 4 & 3 & 5 & 0 \\
            10 & 13 & 5 & 9 & 0 \\
            11 & 10 & 4 & 8 & 0 \\
            12 & 8 & 27 & 0 & 0 \\
        \end{tabular}
    \end{table}

    \begin{table}[htbp]
        \centering
        \caption{Prawn Cracker sample info}
        \label{table:crackersampleinfo}
        \begin{tabular}{>{\bfseries}ccccc}
            \hline
            Sample (no) & Prawn percentage\\
            \hline
            1 & 12\% \\
            2 & 17.6\% \\
            3 & 15\% \\
            4 & 12\% \\
            5 & 12\% \\
            6 & 18\% \\
            7 & Unknown \\
            8 & 18\% \\
            9 & Unknown \\
            10 & Unknown \\
            11 & 20\% \\
            12 & 16\% \\
        \end{tabular}
    \end{table}

    \clearpage
    \bibliography{2020-02-08-fibrous-microplastic-particles-in-prawn-crackers}
    \bibliographystyle{plain}
\end{document}
